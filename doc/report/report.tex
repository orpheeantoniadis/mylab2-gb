\documentclass[a4paper]{article}
\usepackage[francais]{babel}
\usepackage{fontspec}
\usepackage{enumitem}
\usepackage{authblk}
\usepackage{minted}
\setlength{\parindent}{0pt}
\usepackage[left=2.5cm,top=2.5cm,right=2.5cm,bottom=2.5cm]{geometry}

\title{MyLab2 GB \protect\\ Emulateur de Gameboy sur MyLab2}
\author{Orphée Antoniadis}
\affil{\small Projet de semestre - Prof. Fabien Vannel}
\affil{\small Hepia ITI 3\up{ème} année}
\date{Semestre d'automne 2017-2018}

\begin{document}
\maketitle

\newpage
\setcounter{tocdepth}{4}
\tableofcontents
\newpage

%%%%%%%%%%%%%%%%%%%%%%%%%%%%%%%%%%%%%%%%%%%%%%%%%%%%%%%%%%%%%%%%%%
%%%%%%%%%%%%%%%%%%%%%%%%%%%%%%%%%%%%%%%%%%%%%%%%%%%%%%%%%%%%%%%%%%

\section{Introduction}
\subsection{Objectif}
L'objectif principal de ce projet était de développer un émulateur de la console
de jeux vidéos de Nintendo appelée Gameboy pour la carte de d'extension MyLab2
développée à l'hepia. Un emulateur est un logiciel permettant d'imiter le comportement
physique d'un matériel informatique. Le but ici est donc d'imiter le comportement
du processeur de la Gameboy avec tous ces périphériques. \\

Il a fallut dans un premier temps comprendre le fonctionnement de la Gameboy. 
Le projet a donc commencé par un travail de recherche et de documentation. Etant 
donné que Nintendo n'a jamais rendu public la datasheet de sa console, toutes les 
informations récoltées ont été obtenues après un travail de reverse engineering 
effectuée par la communauté. Heureusement, la communauté de développeurs d'émulateurs 
est très active, de nombreux forums et blogs existent ce qui a simplifié la recherche
d'informations. Toutes les sources que j'ai trouvé seront listées à la fin de se 
document et pourront faire office de base de donnée complète pour quiconque voulant 
en apprendre plus sur la console ou bien même voulant développer son propre émulateur.

%%%%%%%%%%%%%%%%%%%%%%%%%%%%%%%%%%%%%%%%%%%%%%%%%%%%%%%%%%%%%%%%%%

\subsection{Méthode}

%%%%%%%%%%%%%%%%%%%%%%%%%%%%%%%%%%%%%%%%%%%%%%%%%%%%%%%%%%%%%%%%%%

\subsection{Schéma}

%%%%%%%%%%%%%%%%%%%%%%%%%%%%%%%%%%%%%%%%%%%%%%%%%%%%%%%%%%%%%%%%%%
%%%%%%%%%%%%%%%%%%%%%%%%%%%%%%%%%%%%%%%%%%%%%%%%%%%%%%%%%%%%%%%%%%

\section{MyLab2}
\subsection{Introduction}
La MyLab2 est une carte d'extension pour les kits de développement des microcontrolleurs
LPC1769 et LPC4337 de NXP. Cette carte a été développée à hepia, au laboratoire
de systèmes numériques. Elle vient se superposer aux cartes de NXP via des connecteurs
broches. \\

Le LPC1769 de NXP est un microcontrolleur ARM Cortex-M3 ayant une fréquence d'horloge
pouvant aller jusqu'à 120MHz. Ce dernier propose une mémoire flash de 512kB (mais
seulement la moitié est programmable avec la version gratuite de LPCXpresso) ainsi
qu'une SRAM de 64kB. A noter que les 64kB de SRAM sont discontinus et qu'il y a en
réalité deux banques de RAM de 32kB chacune. 
Le LPC4337 de NXP est un microcontrolleur ARM Cortex-M4 ayant une fréquence d'horloge 
pouvant aller jusqu'à 204MHz mais inclut aussi un coprocesseur ARM Cortex-M0 pouvant
aller à la même fréquence d'horloge. La SRAM de ce dernier fait 136kB. \\

Pour ce projet, le microcontrolleur utilisé sera le LPC1769 pour un soucis pratique.
En effet, nous avons déjà eu à utiliser cette carte pour le cours de microcontrolleurs
et périphériques de deuxième année ce qui fait que j'avais déjà une grande partie de
la librairie de gestion des périphériques de la carte d'extension prête. C'était un
gain de temps non négligeable qui m'a permis de me concentrer sur l'émulation de
la Gameboy. \\

Le projet pourrait tout de même être repris et porté sur le LPC4337 afin d'avoir de
meilleures performances étant donné que sa fréquence d'horloge est deux fois plus
grande, qu'il dispose de deux processeurs permettant une parallélisation des taches
et qu'il y a plus de RAM.

%%%%%%%%%%%%%%%%%%%%%%%%%%%%%%%%%%%%%%%%%%%%%%%%%%%%%%%%%%%%%%%%%%

\subsection{Pérphériques utilisés}
\subsubsection{Boutons}
Les boutons A et B de la MyLab2 sont de simples pins GPIO. Le LPC1769 propose
des interruptions (EINT3) sur les ports 0 et 2 du GPIO. Ici, le bouton A est sur la pin
10 du port 2 (P2.10) et le bouton B est sur la pin 19 du port 0 (P0.19). L'interruption
peut se faire sur le flanc montant ou descendant (ou les deux). Pour l'activer,
il faut mettre le bit correspondant à 1 dans le registre \mintinline{c}{IntEnF}
(flanc descendant) ou \mintinline{c}{IntEnR} (flanc montant). Lors d'une interruption,
il faut regarder laquelle a eu lieu dans le registre \mintinline{c}{IntStatF}/\mintinline{c}{IntStatR}
puis la quittancer en mettant le bon bit à 1 dans le registre \mintinline{c}{IntClr}.

\subsubsection{Joystick}
Le joystick de la MyLab2 est aussi relié à une pin GPIO mais contrairement aux
boutons A et B, il est relié au port 1 du GPIO. Il n'y a pas d'interruptions
sur le port 1 du GPIO, il faut donc vérifier l'état des pins du joystick de manière
régulière. Pour se faire, un timer sera utilisé qui provoquera une interruption
(TIMER0) toutes les 10ms. La routine d'interruption appelle la fonction \mintinline{c}{joystick_handler}
qui prend comme argument une fonction de callback (un pointeur sur une fonction
qui sera appelée si une des directions du joystick est appuyée), l'argument de cette 
dernière, et le mode de vérification (\mintinline{c}{POLLING} ou \mintinline{c}{TRIGGER}). 
En mode \mintinline{c}{POLLING}, la fonction de callback sera appelée qu'une seule
fois si le joystick est maintenu appuyé contrairement au mode \mintinline{c}{TRIGGER}
où la fonction sera appelée tant que le joystick n'est pas relâché. La fonction de
callback doit avoir comme prototype:

\inputminted[breaklines,breaksymbol=,linenos,frame=single,stepnumber=5,tabsize=2,firstline=14,lastline=14]
{c}{../../workspace/mylab2-gb/src/controls.c}

avec \mintinline{c}{pos} qui est la position du joystick, \mintinline{c}{edge} qui
est le flanc (montant ou descendant) et \mintinline{c}{arg} qui contient l'argument
de la fonction (peut être à \mintinline{c}{NULL}).

\subsubsection{Ecran LCD}
La MyLab2 utilise le le bus de communication SPI pour communiquer avec l’écran. 
Afin d'initialiser le SPI, il faut d'abord sélectionner les bonnes pins dans les 
registres appropriés (\mintinline{c}{PINSEL0} et \mintinline{c}{PINSEL1}). En effet, 
les pins sont configurées par défaut sur des pins GPIO. Pour le LPC1769 ce sont 
donc les bits 31:30 de \mintinline{c}{PINSEL0} et 5:0 de \mintinline{c}{PINSEL1} 
qu'il faut modifier. \\

Il faut ensuite fixer la valeur de la fréquence de transmission. Un registre de configuration du
SPI le permet. C'est le registre \mintinline{c}{SPCCR}. La valeur
donné à ce registre va diviser la valeur de l'horloge du SPI. Pour atteindre
une fréquence de transmission maximale, il faut mettre l'horloge du SPI à 100MHz
en modifiant le registre \mintinline{c}{PCLKSEL0} puis fixer la valeur du registre
\mintinline{c}{SPCCR} à 10 pour avoir une fréquence de transmission de 10MHz ($\frac{100}{10}$).
Pour finir il faut juste activer la communication SPI en la mettant en master mode.
Ceci se fait dans le registre \mintinline{c}{SPCR} (bit 5 à 1). \\

L'écriture et la lecture du bus SPI se fait à l'aide du même registre pour le
LPC1769. Ce registre est le registre  \mintinline{c}{SPDR}. Pour écrire,
il faut fixer la valeur du registre aux données à envoyer puis attendre que le flag
de confirmation d'envoi soit mis à 1. Ce flag est le bit 7 du registre \mintinline{c}{SPSR}.
\newline
Pour venir lire sur le bus SPI, il faut d'abord envoyer 0xFF (donc en utilisant
la fonction d'écriture) puis venir lire sur ce même registre (buffer bi-directionnel). \\

L'écran LCD accepte 2 types de données, les instructions et les arguments
d'instruction. La pin DC (pin GPIO 1.16) permet de contrôler ce qui va être envoyé.
Le DC est à 0 lors de l'envoi d'une instruction et à 1 pour l'envoi d'un argument.
La liste complète des instructions est disponible dans la documentation de l'écran
LCD. Pour ce projet, seulement 3 de ces instructions sont utilisées. Celle pour séléctionner
la colonne d'écriture en pixels (instruction 0x2A), celle pour séléctionner la ligne
d'écriture en pixels (instruction 0x2B) et enfin, celle pour écrire dans la mémoire
de l'écran (instruction 0x2C). La mémoire est l'ensemble des données qui composent
l'intégralité des pixels de l'écran. Ici l'écran est de $240 \times 320$ et un pixel
est codé sur 16 bits donc la mémoire serait ces $240 \times 320 \times 16$ bits.
Pour allumer un pixel sur l'écran, il faut d'abord séléctionner la zone d'écriture
en utilisant les instructions 0x2A et 0x2B, puis il faut écrire les données relatives 
à la couleur du pixel désirée en utilisant l'instruction 0x2C).

\subsubsection{Dalle tactile}


%%%%%%%%%%%%%%%%%%%%%%%%%%%%%%%%%%%%%%%%%%%%%%%%%%%%%%%%%%%%%%%%%%
%%%%%%%%%%%%%%%%%%%%%%%%%%%%%%%%%%%%%%%%%%%%%%%%%%%%%%%%%%%%%%%%%%

\section{Gameboy}
\subsection{Introduction}

%%%%%%%%%%%%%%%%%%%%%%%%%%%%%%%%%%%%%%%%%%%%%%%%%%%%%%%%%%%%%%%%%%

\subsection{CPU}
\subsubsection{Architecture}

\subsubsection{Registres}

\subsubsection{Instructions}

%%%%%%%%%%%%%%%%%%%%%%%%%%%%%%%%%%%%%%%%%%%%%%%%%%%%%%%%%%%%%%%%%%

\subsection{Mémoire}
\subsubsection{Memory map}

\subsubsection{Cartouche}

\subsubsection{Bootstrap}

\subsubsection{DMA}

%%%%%%%%%%%%%%%%%%%%%%%%%%%%%%%%%%%%%%%%%%%%%%%%%%%%%%%%%%%%%%%%%%

\subsection{Interruptions}
\subsubsection{Activer une interruption}

\subsubsection{Demander une interruption}

\subsubsection{Traiter une interruption}

\subsubsection{Comportement de HALT}

%%%%%%%%%%%%%%%%%%%%%%%%%%%%%%%%%%%%%%%%%%%%%%%%%%%%%%%%%%%%%%%%%%

\subsection{Timer}
\subsubsection{Timer controller}

\subsubsection{Divider register}


%%%%%%%%%%%%%%%%%%%%%%%%%%%%%%%%%%%%%%%%%%%%%%%%%%%%%%%%%%%%%%%%%%

\subsection{GPU}
\subsubsection{Contrôleur LCD}

\subsubsection{Affichage}

%%%%%%%%%%%%%%%%%%%%%%%%%%%%%%%%%%%%%%%%%%%%%%%%%%%%%%%%%%%%%%%%%%

\subsection{Entrées}

%%%%%%%%%%%%%%%%%%%%%%%%%%%%%%%%%%%%%%%%%%%%%%%%%%%%%%%%%%%%%%%%%%
%%%%%%%%%%%%%%%%%%%%%%%%%%%%%%%%%%%%%%%%%%%%%%%%%%%%%%%%%%%%%%%%%%

\section{Émulateur}

%%%%%%%%%%%%%%%%%%%%%%%%%%%%%%%%%%%%%%%%%%%%%%%%%%%%%%%%%%%%%%%%%%
%%%%%%%%%%%%%%%%%%%%%%%%%%%%%%%%%%%%%%%%%%%%%%%%%%%%%%%%%%%%%%%%%%

\section{Conclusion}

\end{document}
